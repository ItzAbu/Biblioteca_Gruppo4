\documentclass[a4paper,12pt]{article}

% Pacchetti utili
\usepackage[utf8]{inputenc}  % Codifica UTF-8
\usepackage[italian]{babel} % Lingua italiana
\usepackage{amsmath, amssymb} % Simboli matematici
\usepackage{graphicx} % Per inserire immagini
\usepackage{hyperref} % Link cliccabili
\usepackage{listings} % Codice sorgente
\usepackage{xcolor} % Colori personalizzati

% Configurazione per il codice
\lstset{
  language=C#, % Cambia con il linguaggio che hai usato
  basicstyle=\ttfamily\small,
  keywordstyle=\color{blue},
  commentstyle=\color{gray},
  stringstyle=\color{red},
  numbers=left,
  numberstyle=\tiny\color{gray},
  frame=single,
  breaklines=true,
  captionpos=b,
  showspaces=false,
  showstringspaces=false
}

\title{Relazione del Progetto di Informatica}
\author{Il tuo nome}
\date{\today}

\begin{document}

\maketitle

\tableofcontents % Genera un indice dei contenuti

\newpage

\section{Introduzione}
In questa relazione verrà descritto il progetto di informatica svolto, specificando gli obiettivi, le tecnologie utilizzate e i risultati ottenuti.

\section{Obiettivi del Progetto}
Descrivi qui gli obiettivi principali del progetto. Per esempio:
\begin{itemize}
    \item Sviluppare un'applicazione/software per \textit{[specifica scopo]}.
    \item Implementare algoritmi di \textit{[specifica algoritmi]}.
    \item Testare e ottimizzare il sistema per \textit{[specifica uso]}.
\end{itemize}

\section{Strumenti e Tecnologie}
Elenca gli strumenti utilizzati, ad esempio:
\begin{itemize}
    \item Linguaggi di programmazione: Python, Java, ecc.
    \item Librerie o framework: NumPy, Flask, ecc.
    \item Ambiente di sviluppo: Visual Studio Code, PyCharm, ecc.
\end{itemize}

\section{Descrizione del Progetto}
Descrivi il progetto in dettaglio, ad esempio:
\begin{itemize}
    \item Architettura del sistema.
    \item Funzionamento del codice.
    \item Problemi riscontrati e soluzioni adottate.
\end{itemize}

\subsection{Struttura del Codice}
Inserisci una breve spiegazione del codice e qualche esempio:

\begin{lstlisting}[caption=Esempio di codice]
def esempio_funzione():
    print("Ciao, mondo!")
\end{lstlisting}

\section{Risultati}
Descrivi i risultati ottenuti:
\begin{itemize}
    \item Prestazioni del sistema.
    \item Test effettuati.
    \item Confronto con gli obiettivi iniziali.
\end{itemize}

\section{Conclusioni}
Riassumi brevemente l'esperienza, i punti di forza e le possibili migliorie future.

\section*{Riferimenti}
Inserisci eventuali fonti o riferimenti bibliografici.

\end{document}
