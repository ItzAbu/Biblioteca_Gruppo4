\documentclass[a4paper,12pt]{article}

% Pacchetti utili
\usepackage[utf8]{inputenc}  % Codifica UTF-8
\usepackage[italian]{babel} % Lingua italiana
\usepackage{amsmath, amssymb} % Simboli matematici
\usepackage{graphicx} % Per inserire immagini
\usepackage{hyperref} % Link cliccabili
\usepackage{listings} % Codice sorgente
\usepackage{xcolor} % Colori personalizzati

\usepackage{xcolor}
\definecolor{darkgreen}{rgb}{0.0, 0.5, 0.0} % Verde scuro personalizzato


% Configurazione per il codice
\lstset{
  language=[Sharp]C, % Linguaggio C#
  basicstyle=\ttfamily\scriptsize,
  keywordstyle=\color{blue},
  commentstyle=\color{gray},
  stringstyle=\color{red},
  numbers=left,
  numberstyle=\tiny\color{gray},
  frame=single,
  breaklines=true,
  captionpos=b,
  showspaces=false,
  showstringspaces=false,
  morekeywords={async, await, var, dynamic, get, set, value}, % Aggiungi parole chiave specifiche di C#
}

\usepackage[a4paper, top=1.5cm, bottom=1.5cm, left=1cm, right=1cm]{geometry}



\title{Relazione del Progetto di Informatica}
\author{Abu Shahid Islam \& Giovanni Sebastiani}
\date{\today}

\begin{document}

\maketitle

\newpage

\tableofcontents % Genera un indice dei contenuti

\newpage

\section{Struttura} 
Il programma è diviso in 2 \textbf{Menu} principali:
\begin{itemize}
    \subsection{Menu Principale}
        \item Qui l'utente può scegliere tra 9 opzioni: \begin{itemize}
        \item \textbf{Aggiungi un libro}
        \item \textbf{Visualizza i libri}
        \item \textbf{Rimuovi un libro}
        \item \textbf{Cerca un libro per autore}
        \item \textbf{Libro più costoso}
        \item \textbf{Ordina libri in base al prezzo}
        \item \textbf{Visualizza per fascia di prezzo}
        \item \textbf{Prestiti}
        \item \textbf{Esci}
    \end{itemize}
    \subsection{Prestiti}
    \item Qui l'utente può scegliere tra 4 opzioni: \begin{itemize}
        \item \textbf{Prendi in prestito}
        \item \textbf{Restituisci il prestito}
        \item \textbf{Visualizza i libri in prestito}
        \item \textbf{Esci}
    \end{itemize}.
    \item Questo programma anche se fatto con Array statici (Array a dimensione fissa) dá la possibilità all'utente illimitati libri
\end{itemize}


\newpage


\section{Menu Principale}
\subsection{Gestione Array}
Funzionamento del Menu Principale:
\begin{itemize}
    \item Chiamato \textbf{Program.cs} contiene il \textit{main}.
    \item Il main contiene tutti gli array per la gestione della libreria
    \item \begin{lstlisting}[caption=Array della libreria]
string[] Titoli = new string[2];
string[] Autori = new string[2];
double[] prezzo = new double[2];
string[] categoria = new string[2];
string[] casa_editrice = new string[2];
int[] copie = new int[2];
        \end{lstlisting}
    \item Contiene gli array per la gestione dei prestiti
    \item \begin{lstlisting}[caption=Array dei prestiti]
int prestiti = 0;
string[] libri_prestito = new string[2];
string[] utenti_prestito_nome = new string[2];
string[] utenti_prestito_cognome = new string[2];
string[] giorno_preso = new string[2];
string[] tempo_trattenuto = new string[2];
int[] codice_prestito = new int[2];
int retuns = 0;
    \end{lstlisting}

    \item Alcune funzioni necessitano il numero di libri presenti nella libreria
    \item \begin{lstlisting}[caption=Funzione per il numero di libri]
int libri_unici = 0;
int libri_unici_copia = 0;
    \end{lstlisting}
\end{itemize}

\newpage

\subsection{Visual}
\begin{itemize}
    \item Tutta la parte visuale è gestita da un menu comandabile con le freccie \textbf{\textit{MenuArrow.cs}}. \newline Questa funzione ha come \textcolor{blue}{\textit{return}} la scelta dell'utente.
    \item In base alla scelta viene scelta la funzione da eseguire.
    \item \begin{lstlisting}[caption=Funzione per il menu]
        //Menu
while (true)
{
    Console.Clear();
       
    //Menu con le freccie
    int scelta = MenuArrow.menuArrow();
       
    switch (scelta)
    {
        case 1:
                    
            AddBook.addBook(ref Titoli, ref Autori, ref prezzo, ref categoria, ref casa_editrice, ref copie, ref libri_unici, ref libri_unici_copia);
            break;
       
        case 2:
                    
            ShowBook.showBook(Titoli, copie);
            break;
       
        case 3:
                    
            RemoveBook.removeBook(ref Titoli, ref Autori, ref prezzo, ref categoria, ref casa_editrice, ref copie, ref libri_unici, ref libri_unici_copia);
            break;
       
        case 4:
                    
            SearchByAuthor.searchByAuthor(Titoli, Autori, copie);
            break;
        case 5:
                    
            MostExpeBook.mostExpeBook(Titoli, prezzo);
            break;
       
        case 6:
       
            Reorder.reorder(ref Titoli, ref Autori, ref prezzo, ref categoria, ref casa_editrice, ref copie);
            break;
       
        case 7:
       
            PriceRange.priceRange(Titoli, prezzo, copie);
            break;
       
        case 8:
                    
            MenuPrestiti.menuPrestiti(Titoli, Autori, prezzo, categoria, casa_editrice, copie, ref libri_unici_copia , ref prestiti, ref libri_prestito, 
                ref utenti_prestito_nome, ref utenti_prestito_cognome, ref giorno_preso, ref tempo_trattenuto, ref codice_prestito, ref retuns);
            break;
       
        case 9:
       
            Console.Clear();
            Console.WriteLine("Arrivederci!");
            Console.ReadKey();
       
            return;
    }
       
}
        
    \end{lstlisting}
\end{itemize}

\newpage

\subsubsection{MenuArrow.cs}
Questa funzione permette di navigare nel menu con le frecce.
Le frecce funzionano tramite un array \textcolor{blue}{strings} che contiene le scelte del menu.
Un intero \textcolor{blue}{pos} che indica la posizione attuale nel menu;
e leggendo il tasto premuto, se è una freccia, cambia la posizione.


Nel seguente codice si può vedere come funziona la chat colorata e la freccia che indica la scelta dell'utente.
\begin{lstlisting}[caption=MenuArrow.cs]
    //Stampa il menu
static public void Menu(int pos)
{
    // Menu
    string[] strings =
    {
    "Inserisci un libro",
    "Consulta tutti i libri della biblioteca",
    "Rimuovi un libro",
    "Ricerca libri per autore",
    "Stampa libro piu costoso",
    "Ordina i libri in base al prezzo",
    "Stampa i libri per fascia di prezzo",
    "Prestito",
    "Esci"
    };

    // Calcola la larghezza della console

    int consoleWidth = Console.WindowWidth; 

    for (int i = 0; i < strings.Length; i++)
    {
        // Calcola il numero di spazi iniziali per centrare la stringa
        int padding = (consoleWidth - strings[i].Length) / 2;

        // Assicurati che il padding sia almeno 0
        if(padding < 0)
        {
            padding = 0;
        }


        // Colore verde per l'opzione selezionata
        if (i == pos - 1)
        {
            Console.ForegroundColor = ConsoleColor.Green;
            Console.WriteLine(new string(' ', padding) + $"> {strings[i]}");
        }
        else
        {
            Console.ForegroundColor = ConsoleColor.White;
            Console.WriteLine(new string(' ', padding) + $"  {strings[i]}");
        }
    }

    // Resetta il colore
    Console.ResetColor();
}
\end{lstlisting}

\begin{itemize}
    \item \textcolor{blue}{\textbf{Console.WindowWidth}} restituisce la larghezza della console.
    \item \textcolor{blue}{\textbf{Console.ForegroundColor}} cambia il colore del testo.
    \item \textcolor{blue}{\textbf{Console.ResetColor}} resetta il colore del testo.
    \item \textcolor{blue}{\textbf{padding}} calcola il numero di spazi iniziali per centrare la stringa.
    \item \textcolor{blue}{\textbf{new string(' ', padding)}} crea una stringa di spazi in base al padding.
\end{itemize}



\newpage

Invece qui si può vedere come funziona la freccia e la scelta dell'utente.

\begin{lstlisting} [caption=MenuArrow.cs]
//Stampa il menu con la freccia

static public int menuArrow()
{
    //Leggo il tasto premuto
    ConsoleKeyInfo key;
    int pos = 1;
    do
    {
        Console.Clear();
        Console.WriteLine();
        Bibleoteca.Logo();
        Menu(pos);
        Console.WriteLine();
        key = Console.ReadKey();
        Se premo freccia giu incremento la posizione quindi va all opzione dopo
        if (key.Key == ConsoleKey.DownArrow)
        {
            if (pos < 9)
            {
                pos++;
            }
        }
        //Se premo freccia su decremento la posizione quindi va all`opzione prima
        else if (key.Key == ConsoleKey.UpArrow)
        {
            if (pos > 1)
            {
                pos--;
            }
        }
    } while (key.Key != ConsoleKey.Enter);
    return pos;
}
\end{lstlisting}

\begin{itemize}
    \item \textcolor{blue}{\textbf{ConsoleKeyInfo}} legge il tasto premuto.
    \item \textcolor{blue}{\textbf{Console.Clear}} pulisce la console.
    \item \textcolor{blue}{\textbf{Console.ReadKey}} legge il tasto premuto.
    \item \textcolor{blue}{\textbf{ConsoleKey.DownArrow}} controlla se è stata premuta la freccia giù.
    \item \textcolor{blue}{\textbf{ConsoleKey.UpArrow}} controlla se è stata premuta la freccia su.
    \item \textcolor{blue}{\textbf{ConsoleKey.Enter}} controlla se è stato premuto il tasto invio.
    \item \textcolor{blue}{\textbf{pos}} indica la scelta corrente.
\end{itemize}


\newpage

\subsection{AddBook.cs}
\begin{itemize}
    \item Questa funzione permette di aggiungere un libro alla libreria.
    \item 
    \subsubsection{Step 1}
        Primo step è controllare se ci sono posti disponibili dove mettere i libri.
        In caso contrario, il programma crea un nuovo spazio tramite \textcolor{darkgreen}{\textit{Array}}\textcolor{yellow}{\textit{.Resize}}.

\begin{lstlisting}[caption=Controllo spazio disponibile]
Console.Clear();
//Se abbiamo raggiunto il massimo
if (libri_unici == Titoli.Length)
{
    //Resize degli array
                    
    Array.Resize(ref Titoli, Titoli.Length + 1);
    Array.Resize(ref Autori, Autori.Length + 1);
    Array.Resize(ref prezzo, prezzo.Length + 1);
    Array.Resize(ref categoria, categoria.Length + 1);
    Array.Resize(ref casa_editrice, casa_editrice.Length + 1);
    Array.Resize(ref copie, copie.Length + 1);                 
}
\end{lstlisting}

    \newpage

    \subsubsection{Step 2}
    
    \item Inserimento dei dati del libro.
    \item Il programma controlla all'inserimento se il libro è già presente.
    \item in caso sia un libro nuovo chiede le altre informazioni.
    \item Se avviene qualche errore, Es. inserimento di una \textcolor{blue}{stringa} al posto di un \textcolor{blue}{double}, il programma annulla l'inserimento.
        
    
\end{itemize}

\begin{lstlisting}[caption=Inserimento dei dati]
Console.WriteLine("Inserisci il titolo del libro");
string titolo = Console.ReadLine();

    //Libro gia presente o meno

for (int i = 0; i < Titoli.Length; i++)
{
    if (Titoli[i] == null)
    {
        continue;
    }
    if (ToLower.ToLowerString(Titoli[i]) == ToLower.ToLowerString(titolo))
    {
        copie[i]++;
        Console.WriteLine("Il libro e gia presente nei nostri archivi, non servono le altre informazioni");
        Console.ReadKey();
        return;
    }
}

try{
    Titoli[libri_unici] = titolo;

    Console.WriteLine("Inserisci l'autore del libro");
    Autori[libri_unici] = Console.ReadLine();

    Console.WriteLine("Inserisci il prezzo");
    prezzo[libri_unici] = double.Parse(Console.ReadLine());

    Console.WriteLine("Inserisci la categoria");
    categoria[libri_unici] = Console.ReadLine();

    Console.WriteLine("Inserisci la casa editrice");
    casa_editrice[libri_unici] = Console.ReadLine();

    copie[libri_unici] = 1;
}
catch (Exception e)
{
    Console.WriteLine("Errore, Qualcosa e andato storto");
    Console.ReadKey();
    Array.Resize(ref Titoli, Titoli.Length - 1);
    Array.Resize(ref Autori, Autori.Length - 1);
    Array.Resize(ref prezzo, prezzo.Length - 1);
    Array.Resize(ref categoria, categoria.Length - 1);
    Array.Resize(ref casa_editrice, casa_editrice.Length - 1);
    Array.Resize(ref copie, copie.Length - 1);
    return;
}
\end{lstlisting}

\begin{itemize}
    \item \textcolor{blue}{\textbf{Array.Resize}} aumenta/diminuisce la dimensione dell'array.
\end{itemize}

\newpage

\subsubsection{Step 3}
\begin{itemize}
    \item Ultimo step è la stampa dei dati inseriti
\end{itemize}
\begin{lstlisting}[caption={Stampa dei dati inseriti}]
Console.WriteLine("Libro aggiunto con successo");
Console.ReadKey();

string[] strings =
{
    $"Libro: ",
    Titoli[libri_unici],
    $"Autore: ",
    Autori[libri_unici],
    $"Prezzo: ",
    prezzo[libri_unici].ToString(),
    $"Categoria: ",
    categoria[libri_unici],
    $"Casa Editrice",
    casa_editrice [libri_unici],
        
};

Console.Clear();
int padding = 15;

for(int i=0; i < strings.Length; i+=2)
{
        
    Console.Write(new string(' ', padding) + strings[i]);
    int padding2 = Console.WindowHeight - (padding - strings[i].Length)/2;

    Console.Write(new string(' ', padding2-i) + strings[i+1]);

    if(i == 4)
    {
        Console.Write(" Eur");
    }
    Console.WriteLine();
}

//Aumento libri unici
libri_unici++;
libri_unici_copia++;

Console.ReadKey();
    
\end{lstlisting}


\newpage

\subsection{ShowBook.cs}
\subsubsection{Step 1}
Lo scopo di questa funzione è di visualizzare tutti i libri presenti nella libreria.

Questa funzione mostra un menu con tutti i libro, e alla selezione di ogni libro, mostra le informazioni di quel libro.
Il modo di funzionare è simile al menu principale, ma con l'obbiettivo di visualizzare i libri.

\begin{lstlisting}
public static void Menu(int pos, string[] Titoli) 
{
    Console.Clear();
    Console.WriteLine("Indice Libri");

    for(int i=0; i<Titoli.Length; i++)
    {
        if (i == pos-1)
        {
            Console.ForegroundColor = ConsoleColor.Green;
            Console.WriteLine($"> [{i+1}] {Titoli[i]}");
            Console.ResetColor();
        }
        else
        {
            Console.WriteLine($"  [{i+1}] {Titoli[i]}");
        }
    }

}

public static void showBook(string[] Titoli, int[] copie, string[] Autori, double[] prezzo, string[] categoria, string[] casa_editrice)
{
            
    ConsoleKeyInfo key;
    int pos = 1;
    do{
        Menu(pos, Titoli);
        key = Console.ReadKey();

        if(key.Key == ConsoleKey.DownArrow)
        {
            if(pos == Titoli.Length)
            {
                pos = 1;
            }
            else
            {
                pos++;
            }
        }
        else if(key.Key == ConsoleKey.UpArrow)
        {
            if(pos == 1)
            {
                pos = Titoli.Length;
            }
            else
            {
                pos--;
            }
        }
    }while(key.Key != ConsoleKey.Enter);

    Console.Clear();

    //Stampa libri
    Console.WriteLine("Titolo: " + Titoli[pos-1]);
    Console.WriteLine("Autore: " + Autori[pos-1]);
    Console.WriteLine("Prezzo: " + prezzo[pos-1] + "Eur");
    Console.WriteLine("Categoria: " + categoria[pos-1]);
    Console.WriteLine("Casa Editrice: " + casa_editrice[pos-1]);
    Console.WriteLine("Copie: " + copie[pos-1]);

    Console.WriteLine("Premi un tasto per tornare al menu");
    Console.ReadKey();

    return;
}

\end{lstlisting}

\subsection{RemoveBook.cs}
\subsubsection{Step 1}
Lo scopo di questa funzione è di rimuovere un libro dalla libreria.
Funziona con lo stesso identico schema di Menu visto precedentemente.
Ogni libro ha un numero associato, e alla selezione di quel numero, il libro viene rimosso.
In caso ci sia piú di una copia elimina prima una copia, e poi il libro.

\begin{lstlisting}
public static void removeBook(ref string[] Titoli, ref string[] Autori, ref double[] prezzo, ref string[] categoria, ref string[] casa_editrice, ref int[] copie, ref int libri_unici, ref int libri_unici_copia)
{
    ConsoleKeyInfo key;
    int lettura = 1;
    do{
        Menu(lettura, Titoli, copie);
        key = Console.ReadKey();

        if(key.Key == ConsoleKey.DownArrow)
        {
            if(lettura == Titoli.Length)
            {
                lettura = 1;
            }
            else
            {
                lettura++;
            }
        }
        else if(key.Key == ConsoleKey.UpArrow)
        {
            if(lettura == 1)
            {
                lettura = Titoli.Length;
            }
            else
            {
                lettura--;
            }
        }
    }while(key.Key != ConsoleKey.Enter);
}
\end{lstlisting}

\begin{itemize}
    \item \textcolor{blue}{\textbf{Menu}} è la funzione che stampa il menu.
    \item \textcolor{blue}{\textbf{ConsoleKey.DownArrow}} controlla se è stata premuta la freccia giù.
    \item \textcolor{blue}{\textbf{ConsoleKey.UpArrow}} controlla se è stata premuta la freccia su.
    \item \textcolor{blue}{\textbf{ConsoleKey.Enter}} controlla se è stato premuto il tasto invio.
\end{itemize}

\subsubsection{Step 2}

Eliminazione copia

\begin{lstlisting}
    if (copie[lettura - 1] > 1)
    {
        copie[lettura - 1]--;
        Console.Clear();
        Console.WriteLine("Copia eliminata con successo");
        Console.ReadKey();
        Console.Clear();
        return;

    }
\end{lstlisting}

\begin{itemize}
    \item \textcolor{blue}{\textbf{copie[lettura - 1] > 1}} controlla se ci sono piú di una copia.
    \item \textcolor{blue}{\textbf{copie[lettura - 1]--}} elimina una copia.
    \item \textcolor{blue}{\textbf{Console.Clear}} pulisce la console.
\end{itemize}

\newpage

\subsubsection{Step 3}

Eliminazione libro
\begin{lstlisting}
    Titoli[lettura - 1] = "";
    Autori[lettura - 1] = "";
    prezzo[lettura - 1] = 0;
    categoria[lettura - 1] = "";
    casa_editrice[lettura - 1] = "";
    copie[lettura - 1] = 0;

    for (int i = lettura - 1; i < Titoli.Length - 1; i++)
    {
        string temp;
        int tempint;
        double tempdouble;

        temp = Titoli[i + 1];
        Titoli[i] = temp;

        temp = Autori[i + 1];
        Autori[i] = temp;

        tempdouble = prezzo[i + 1];
        prezzo[i] = tempdouble;

        temp = categoria[i + 1];
        categoria[i] = temp;

        temp = casa_editrice[i + 1];
        casa_editrice[i] = temp;

        tempint = copie[i + 1];
        copie[i] = tempint;
    }

    //Resize degli array
    Array.Resize(ref Titoli, Titoli.Length - 1);
    Array.Resize(ref Autori, Autori.Length - 1);
    Array.Resize(ref prezzo, prezzo.Length - 1);
    Array.Resize(ref categoria, categoria.Length - 1);
    Array.Resize(ref casa_editrice, casa_editrice.Length - 1);
    Array.Resize(ref copie, copie.Length - 1);

    libri_unici--;
    libri_unici_copia--;

    Console.Clear();
    Console.WriteLine("Il libro e stato eliminato");
    Console.ReadKey();
    Console.Clear();
\end{lstlisting}

\begin{itemize}
    \item \textcolor{blue}{\textbf{Titoli[lettura - 1] = ""}} elimina il titolo.
    \item \textcolor{blue}{\textbf{Array.Resize}} diminuisce la dimensione dell'array.
    \item \textcolor{blue}{\textbf{libri\_unici--}} diminuisce il numero di libri unici.
    \item \textcolor{blue}{\textbf{libri\_unici\_copia--}} diminuisce il numero di libri unici copia.
\end{itemize}

\newpage

\subsection{SearchByAuthor.cs}
\subsubsection{Step 1}
Menu (guardare sopra per spiegazione) e ricerca per autore.

Il seguente codice, crea un array con tutti gli autori presenti, e permette di scegliere un autore.

\begin{lstlisting}
static public void searchByAuthor(string[] Titoli, string[] Autori, int[] copie)
{
    Console.Clear();
    Console.WriteLine("Autori Presenti");

    string[] autori_unici = new string[1];

    //Metto dentro l`array tutti i autori 
    for (int i = 0; i < Autori.Length; i++)
    {
        if(Autori[i] == null) {
            continue;
        }
        //Controllo che non ci siano doppioni
        bool doppione = false;
        for (int j = 0; j < autori_unici.Length; j++)
        {
            if (autori_unici[j] == null)
            {
                autori_unici[j] = Autori[i];
                break;
            }
                    
            if (ToLower.ToLowerString(Autori[i]) == ToLower.ToLowerString(autori_unici[j]))
            {
                doppione = true;
                break;
            }
        }

        if (!doppione)
        {
            Array.Resize(ref autori_unici, autori_unici.Length + 1);
            autori_unici[autori_unici.Length - 1] = Autori[i];
        }

    }

    if(Autori[0] == null)
    {
        Console.WriteLine("Nessun autore presente");
        Console.ReadKey();
        return;
    }

    
\end{lstlisting}

\begin{itemize}
    \item \textcolor{blue}{\textbf{autori\_unici}} contiene tutti gli autori presenti.
    \item \textcolor{blue}{\textbf{ToLower.ToLowerString}} trasforma la stringa in minuscolo.
    \item \textcolor{blue}{\textbf{Array.Resize}} aumenta la dimensione dell'array.
\end{itemize}

\newpage


\subsubsection{Step 2}
Gestore della scelta dell'autore

\begin{lstlisting}
    //Stampo
    ConsoleKeyInfo key;
    int scelta = 1;
    do{
        Menu(scelta, autori_unici);
        key = Console.ReadKey();

        if(key.Key == ConsoleKey.DownArrow)
        {
            if(scelta == autori_unici.Length)
            {
                scelta = 1;
            }
            else
            {
                scelta++;
            }
        }
        else if(key.Key == ConsoleKey.UpArrow)
        {
            if(scelta == 1)
            {
                scelta = autori_unici.Length;
            }
            else
            {
                scelta--;
            }
        }
    }while(key.Key != ConsoleKey.Enter);

    
\end{lstlisting}

\begin{itemize}
    \item \textcolor{blue}{\textbf{Menu}} è la funzione che stampa il menu.
    \item \textcolor{blue}{\textbf{ConsoleKey.DownArrow}} controlla se è stata premuta la freccia giù.
    \item \textcolor{blue}{\textbf{ConsoleKey.UpArrow}} controlla se è stata premuta la freccia su.
    \item \textcolor{blue}{\textbf{ConsoleKey.Enter}} controlla se è stato premuto il tasto invio.
    \item \textcolor{blue}{\textbf{scelta}} indica la scelta dell'utente.
\end{itemize}

\subsubsection{Step 3}

Stampa dei libri dell'autore scelto, tramite un ciclo \textcolor{blue}{for} che controlla gli autori di tutti i libri e in caso affermativo stampa

\begin{lstlisting}
    //Stampo i libri dell`autore scelto
    Console.Clear();
    Console.WriteLine($"Libri di {autori_unici[scelta - 1]}");
    int pos = 0;
    for (int i = 0; i < Autori.Length; i++)
    {
        if (ToLower.ToLowerString(Autori[i]) == ToLower.ToLowerString(autori_unici[scelta - 1]))
        {
            Console.WriteLine($"[{pos + 1}] {Titoli[i]} \t x{copie[i]}");
            pos++;
        }
                
    }

    Console.ReadKey();
    return;
}
\end{lstlisting}

\newpage

\subsection{MostExpeBook.cs}
\subsubsection{Step 1}

Lo scopo di questa funzione è di trovare il libro piú costoso.
Cerca la posizione del prezzo piu alto e stampa il libro.

\begin{lstlisting}
public static void mostExpeBook(string[] Titoli, double[] prezzo)
{
    Console.Clear();
    double max = -1;
    int pos = 0;

    for (int i = 0; i < prezzo.Length; i++)
    {
        if (prezzo[i] > max)
        {
            max = prezzo[i];
            pos = i;
        }
    }

    Console.WriteLine($"Il libro piu costoso: {Titoli[pos]} \t {prezzo[pos]} Eur");
    Console.ReadKey();
    return;
}
\end{lstlisting}

\newpage

\subsection{Reorder.cs}
\subsubsection{Step 1}
Lo scopo di questa funzione è di ordinare i libri in base al prezzo.
Viene usato un algoritmo di ordinamento \textcolor{blue}{Bubble Sort}.
Si ha una scelta tra ordinamento crescente e decrescente.

Menu
\begin{lstlisting}
    Console.Clear();
    Console.WriteLine("Scegli come riordinare");
    Console.WriteLine("[1] Crescente \t [2] Decrescente");
    int scelta;
    try
    {
        scelta = int.Parse(Console.ReadLine());
    }
    catch (Exception e)
    {
        Console.WriteLine("Errore: Inserire un numero valido");
        Console.ReadKey();
        return;
    }

    if(scelta != 1 && scelta != 2)
    {
        Console.WriteLine("Errore: Inserire un numero valido");
        Console.ReadKey();
        return;
    }
\end{lstlisting}

\subsubsection{Step 2}
Ordinamento crescente 
\begin{lstlisting}
if (scelta == 1)
    {
        for(int i = 0; i < prezzo.Length-1; i++)
        {
            for (int j = i + 1; j < prezzo.Length; j++)
            {
                if (prezzo[i] > prezzo[j])
                {

                    string temp = Titoli[i];
                    Titoli[i] = Titoli[j];
                    Titoli[j] = temp;

                    temp = Autori[i];
                    Autori[i] = Autori[j];
                    Autori[j] = temp;

                    double temp2 = prezzo[i];
                    prezzo[i] = prezzo[j];
                    prezzo[j] = temp2;

                    temp = categoria[i];
                    categoria[i] = categoria[j];
                    categoria[j] = temp;

                    temp = casa_editrice[i];
                    casa_editrice[i] = casa_editrice[j];
                    casa_editrice[j] = temp;

                    int temp3 = copie[i];
                    copie[i] = copie[j];
                    copie[j] = temp3;
                }
            }
        }
    }
\end{lstlisting}

\newpage
Ordinamento Decrescente
\begin{lstlisting}
    if (scelta == 1)
    {
        for(int i = 0; i < prezzo.Length-1; i++)
        {
            for (int j = i + 1; j < prezzo.Length; j++)
            {
                if (prezzo[i] < prezzo[j])
                {

                    string temp = Titoli[i];
                    Titoli[i] = Titoli[j];
                    Titoli[j] = temp;

                    temp = Autori[i];
                    Autori[i] = Autori[j];
                    Autori[j] = temp;

                    double temp2 = prezzo[i];
                    prezzo[i] = prezzo[j];
                    prezzo[j] = temp2;

                    temp = categoria[i];
                    categoria[i] = categoria[j];
                    categoria[j] = temp;

                    temp = casa_editrice[i];
                    casa_editrice[i] = casa_editrice[j];
                    casa_editrice[j] = temp;

                    int temp3 = copie[i];
                    copie[i] = copie[j];
                    copie[j] = temp3;
                }
            }
        }
    }    
\end{lstlisting}

\subsubsection{Step 3}
Stampa dei libri ordinati
\begin{lstlisting}
    Console.WriteLine("Libri riordinati con successo. Premere un tasto per continuare");
    Console.ReadKey();
    Console.Clear();
    Console.WriteLine("Indice  - Titolo - Autore - Prezzo - Categoria - Casa Editrice - Copie");
    for (int  i = 0;  i < Titoli.Length;  i++)
    {
        Console.WriteLine($"{i+1} - {Titoli[i]} - {Autori[i]} - {prezzo[i]} - {categoria[i]} - {casa_editrice[i]} - x{copie[i]}");
    }

    Console.ReadKey();
    return;
\end{lstlisting}

\newpage

\subsection{PriceRange.cs}

\subsubsection{Step 1}
Lo scopo di questa funzione è di visualizzare i libri in base ad una fascia di prezzo.
Viene chiesto all'utente di inserire un prezzo minimo e massimo, e vengono stampati i libri in quella fascia.

\begin{lstlisting}
public static void priceRange(string[] Titoli, double[] prezzo, int[] copie)
{
    Console.Clear();
    Console.WriteLine("Inserisci la fascia di prezzo");
    double min, max;
    try
    {
        Console.WriteLine("Minimo");
        min = double.Parse(Console.ReadLine());
        Console.WriteLine("Massimo");
        max = double.Parse(Console.ReadLine());
    }
    catch (Exception e)
    {
        Console.WriteLine("Errore: Inserire un numero valido");
        Console.ReadKey();
        return;

    }

    if (min > max)
    {
        Console.WriteLine("Errore: Il minimo non puo essere maggiore del massimo");
        Console.ReadKey();
        return;
    }

    for(int i = 0; i < prezzo.Length; i++)
    {
        if (prezzo[i] >= min && prezzo[i] <= max)
        {
            Console.WriteLine($"[{i}] {Titoli[i]} \t {copie[i]} Eur \t x{copie[i]}");
                    
            Console.WriteLine();
        }
    }

    Console.ReadKey();
    return;

}
\end{lstlisting}

\newpage

\section{Prestiti}
\subsection{MenuPrestiti.cs}
\subsubsection{Step 1}
Lo scopo di questa funzione è di gestire i prestiti.
Viene chiesto all'utente di scegliere tra 4 opzioni:
\begin{itemize}
    \item Prendi in prestito
    \item Restituisci il prestito
    \item Visualizza i libri in prestito
    \item Esci
\end{itemize}

Viene tutto salvato negli array citati all'inizio.
Essendo il menu uguale a quello principale, non verrà spiegato.

\begin{lstlisting}
    ConsoleKeyInfo key;
    int pos = 1;
    do
    {
        Console.Clear();
        Console.WriteLine();
        LogoPrestito.logoPrestito();
        Menu(pos);
        key = Console.ReadKey();
        //Se premo freccia giu incremento la posizione quindi va all`opzione dopo
        if (key.Key == ConsoleKey.DownArrow)
        {
            if (pos < 4)
            {
                pos++;
            }
            if(pos == 5)
            {
                pos = 1;
            }
        }
        //Se premo freccia su decremento la posizione quindi va all`opzione prima
        else if (key.Key == ConsoleKey.UpArrow)
        {
            if (pos > 1)
            {
                pos--;
            }
            if (pos == 0)
            {
                pos = 4;
            }
        }
    } while (key.Key != ConsoleKey.Enter);

\end{lstlisting}

\newpage

switch per la scelta dell'utente
\begin{lstlisting}
    switch (pos)
    {
        case 1:

            TakeBorrow.takeBorrow(Titoli,copie, ref libri_prestito, ref prestiti, ref utenti_prestito_nome, ref utenti_prestito_cognome, ref giorno_preso, ref tempo_trattenuto, ref codice_prestito, ref returns);

            break;

        case 2:

            ReturnBook.returnBook(ref libri_prestito,ref utenti_prestito_nome, ref utenti_prestito_cognome, ref giorno_preso, ref tempo_trattenuto, ref codice_prestito, ref returns);

            break;

        case 3:

            ShowBorrowesBooks.showBorrowedBooks(libri_prestito, utenti_prestito_nome, utenti_prestito_cognome, giorno_preso, tempo_trattenuto, codice_prestito, prestiti);

            break;

        case 4:

            Console.Clear();
            Console.WriteLine("Premi per ritornare al menu principale");

            break;

    }
\end{lstlisting}

\newpage

\subsection{TakeBorrow.cs}
\subsubsection{Step 1}
In questo primo step andiamo a rimuovere dall'array dei libri e copie tutti i libri gia in prestito.
Per poi controllare la presenza di libri o meno.

\begin{lstlisting}
    Console.Clear();
    string[] strings = Titoli;
    int[] ints = copie;
            

    for (int i=0; i < libri_prestito.Length; i++)
    {
        for(int j = 0; j < strings.Length; j++)
        {
            if (libri_prestito[i] == null)
            {
                continue;
            }
            if (libri_prestito[i] == strings[j])
            {
                if (ints[j] > 1)
                {
                    ints[j]--;
                }
                else
                {
                    strings[j] = null;
                }
            }
        }
    }

    //controllo se ci sono libri disponibili
    bool flag = false;
    for (int i = 0; i < strings.Length; i++)
    {
        if (strings[i] != null)
        {
            flag = true;
            break;
        }
    }
    if (!flag)
    {
        Console.WriteLine("Non ci sono libri disponibili");
        Console.ReadKey();
        return;
    }
\end{lstlisting}

\newpage

\subsubsection{Step 2}
Come step 2 l'utente seleziona tramite il menu quale libro prendere in prestito

\begin{lstlisting}
    //Stampo i libri disponibili e faccio il menu per il prestito
    ConsoleKeyInfo key;
    int pos = 0;
    do
    {
                
        Menu(strings, pos);
        key = Console.ReadKey();

        //Freccie
        if (key.Key == ConsoleKey.DownArrow)
        {
            pos++;
            if (pos >= strings.Length)
            {
                pos = 0;
            }
            while (strings[pos] == null)
            {
                pos++;
                if (pos >= strings.Length)
                {
                    pos = 0;
                }
            }
                    
        }
        if (key.Key == ConsoleKey.UpArrow)
        {
            pos--;
            if (pos < 0)
            {
                pos = strings.Length - 1;
            }
            while(strings[pos] == null)
            {
                pos--;
                if (pos < 0)
                {
                    pos = strings.Length - 1;
                }
            }

        }
    } while (key.Key != ConsoleKey.Enter);
\end{lstlisting}

\newpage

\subsubsection{Step 3}
Come step 3 l'utente inserisce il proprio nome, cognome e la data di restituzione, e il programma genera un codice prestito.
Il codice infine stampa i dati completi del prestito.

\begin{lstlisting}
    //Aggiungo il libro preso in prestito
    string nome, cognome, giorno_restituisci;

    Console.WriteLine("Inserisci il tuo nome");
    nome = Console.ReadLine();
    Console.WriteLine("Inserisci il tuo cognome");
    cognome = Console.ReadLine();
    Console.WriteLine("Inserisci il giorno in cui devi restituire il libro, Inserisci GG/MM/AAAA");
    giorno_restituisci = Console.ReadLine();



    if (prestiti >= 2)
    {
        Array.Resize(ref libri_prestito, libri_prestito.Length + 1);
        Array.Resize(ref utenti_prestito_nome, utenti_prestito_nome.Length + 1);
        Array.Resize(ref utenti_prestito_cognome, utenti_prestito_cognome.Length + 1);
        Array.Resize(ref giorni_preso, giorni_preso.Length + 1);
        Array.Resize(ref tempo_trattenuto, tempo_trattenuto.Length + 1);
        Array.Resize(ref codice_prestito, codice_prestito.Length + 1);


    }
    libri_prestito[prestiti - returns] = strings[pos ];
    utenti_prestito_nome[prestiti - returns] = nome;
    utenti_prestito_cognome[prestiti - returns] = cognome;
    giorni_preso[prestiti - returns] = DateTime.Now.ToString("dd/MM/yyyy");
    tempo_trattenuto[prestiti - returns] = giorno_restituisci;
    codice_prestito[prestiti - returns] = prestiti;
    prestiti++;

    Console.WriteLine("Libro preso in prestito con successo");

    Console.WriteLine("Libro : " + libri_prestito[prestiti - 1 - returns]);
    Console.WriteLine("Utente : " + utenti_prestito_nome[prestiti - 1 - returns] + " " + utenti_prestito_cognome[prestiti - 1 - returns]);
    Console.WriteLine("Giorno preso : " + giorni_preso[prestiti - 1 - returns]);
    Console.WriteLine("Giorno da restituire : " + tempo_trattenuto[prestiti - 1 - returns]);
    Console.WriteLine("Codice prestito : " + codice_prestito[prestiti - 1 - returns]);

    Console.ReadKey();

    return;
\end{lstlisting}

\newpage

\subsection{ReturnBook.cs}
\subsubsection{Step 1}
In questo primo step l'utente inserisce il proprio codice e il programma controlla la sua validitá.

\begin{lstlisting}
    Console.Clear();
    Console.WriteLine("Inserisci il codice del prestito da restituire");
    int codice;
    try
    {
        codice = int.Parse(Console.ReadLine());
    }
    catch (Exception e)
    {
        Console.WriteLine("Errore: Inserire un numero");
        return;
    }

    int pos;
    //Flag codice non presente
    bool flag = false;
    for (pos = 0; pos < codice_prestito.Length; pos++)
    {
        if (codice_prestito[pos] == codice)
        {
            break;
        }
        if(pos == codice_prestito.Length - 1)
        {
            flag = true;
        }
    }

    if(flag)
    {
        Console.WriteLine("Codice non presente");
        Console.ReadKey();
        return;
    }
\end{lstlisting}
\textcolor{red}{\textbf(Nota: I codici non sono stati descritti in quanto sono molto simili a quelli visti precedentemente)}


\newpage

\subsubsection{Step 2}
Step 2, stampare le informazioni del prestito e chiedere conferma.

\begin{lstlisting}
    //Prestito restituito si/no menu
    ConsoleKeyInfo key;
    int pos3 = 0;
    do
    {
        Console.Clear();

        string[] strings =
        {
        "Nome: " + utenti_prestito_nome[pos],
        "Cognome: " + utenti_prestito_cognome[pos],
        "Libro: " + libro_prestito[pos],
        "Giorno preso: " + giorni_preso[pos],
        "Tempo trattenuto: " + tempo_trattenuto[pos],
        "Codice prestito: " + codice_prestito[pos],
        "Confermi la restituzione? "

        };
        int consoleWidth = Console.WindowWidth;
        for (int i = 0; i < strings.Length; i++)
        {
            int padding = (consoleWidth - strings[i].Length) / 2;
            if (padding < 0)
            {
                padding = 0;
            }
            Console.WriteLine(new string(' ', padding) + $"  {strings[i]}");
        }
        Menu(pos3);
        key = Console.ReadKey();
        if (key.Key == ConsoleKey.LeftArrow)
        {
            if (pos3 == 1)
                pos3--;
                    
        }
        else if (key.Key == ConsoleKey.RightArrow)
        {
            if(pos3 == 0)
                pos3++;
        }
    } while (key.Key != ConsoleKey.Enter);

\end{lstlisting}
\textcolor{red}{\textbf(Nota: I codici non sono stati descritti in quanto sono molto simili a quelli visti precedentemente)}


\newpage

\subsubsection{Step 3}
Data la conferma, il programma rimuove il prestito e stampa un messaggio di conferma.

\begin{lstlisting}
if (pos3 == 1)
    {
        Console.WriteLine("Azione annullata");
        Console.ResetColor();
        Console.ReadKey();
        return;
    }
    else
    {
        //Restituzione libro
        libro_prestito[pos] = null;
        utenti_prestito_nome[pos] = null;
        utenti_prestito_cognome[pos] = null;
        giorni_preso[pos] = null;
        tempo_trattenuto[pos] = null;
        codice_prestito[pos] = 0;

        for (int i = pos; i < libro_prestito.Length - 1; i++)
        {
            libro_prestito[i] = libro_prestito[i + 1];
            utenti_prestito_nome[i] = utenti_prestito_nome[i + 1];
            utenti_prestito_cognome[i] = utenti_prestito_cognome[i + 1];
            giorni_preso[i] = giorni_preso[i + 1];
            tempo_trattenuto[i] = tempo_trattenuto[i + 1];
            codice_prestito[i] = codice_prestito[i + 1];
        }

        Array.Resize(ref libro_prestito, libro_prestito.Length - 1);
        Array.Resize(ref utenti_prestito_nome, utenti_prestito_nome.Length - 1);
        Array.Resize(ref utenti_prestito_cognome, utenti_prestito_cognome.Length - 1);
        Array.Resize(ref giorni_preso, giorni_preso.Length - 1);
        Array.Resize(ref tempo_trattenuto, tempo_trattenuto.Length - 1);
        Array.Resize(ref codice_prestito, codice_prestito.Length - 1);
        returns++;
        Console.WriteLine("Libro restituito con successo");
    }

    Console.ReadKey();
    return;
\end{lstlisting}

\textcolor{red}{\textbf(Nota: I codici non sono stati descritti in quanto sono molto simili a quelli visti precedentemente)}

\newpage

\subsection{ShowBorrowedBooks.cs}
\subsubsection{Step 1}
Lo scopo di questa funzione è di visualizzare i libri in prestito.
Vengono stampadi tutti i prestiti attivi al momento.

\begin{lstlisting}
    Console.Clear();
    Console.WriteLine("Prestiti:");
    for (int i = 0; i < prestiti; i++)
    {
        if(libri_prestito.Length <= i)
        {
            break;
        }
        Console.WriteLine("Codice prestito: " + codice_prestito[i]);
        Console.WriteLine($"\tLibro: {libri_prestito[i]}");
        Console.WriteLine($"\tUtente: {utenti_prestito_nome[i]} {utenti_prestito_cognome[i]}");
        Console.WriteLine($"\tGiorno preso: {giorni_preso[i]}");
        Console.WriteLine($"\tTempo trattenuto: {tempo_trattenuto[i]}");
        Console.WriteLine();
    }
    Console.WriteLine("Premi un tasto per tornare al menu");
    Console.ReadKey();
\end{lstlisting}

\newpage

\section{Utils e Logo}
\subsection{ToLower.cs}
Una funzione gia esistente in C\# che trasforma una stringa in minuscolo, ma per alcuni bug ne abbiamo creato una noi.
\begin{lstlisting}
    public static string ToLowerString(string str)
    {
        //Converte la stringa in minuscolo
        string lower = "";
        for(int i = 0; i < str.Length; i++)
        {
            //Le lettere maiuscole vengono convertite in minuscole
            if (str[i] >= 'A' && str[i] <= 'Z')
            {
                lower += str[i].ToString().ToLower();
            }
            else
            {
                lower += str[i];
            }
                
        }
        return lower;
    }
\end{lstlisting}

\subsection{LogoBiblioteca.cs || LogoPrestito.cs}
Logo della biblioteca e Prestito fatte tramite un sito TopSecret

\textcolor{red}{\tiny{\textbf(Nota: Il codice dei loghi sono reperibili su github!)}}

\begin{figure}[h!]
    \centering
    \includegraphics[width=0.5\textwidth]{biblioteca.png}
    \caption{Logo biblioteca.}
    \label{fig:etichetta}
\end{figure}

\begin{figure}[h!]
    \centering
    \includegraphics[width=0.5\textwidth]{Prestito.png}
    \caption{Logo Prestito.}
    \label{fig:etichetta}
\end{figure}



\end{document}
